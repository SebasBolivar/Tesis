\documentclass[10pt,a4paper]{article}
\usepackage[utf8]{inputenc}
\usepackage[spanish,es-tabla]{babel}
\usepackage{url}
\usepackage{natbib}
\usepackage{hyperref}
\usepackage{amsmath}
\usepackage{amsfonts}
\usepackage{amssymb}
\usepackage{graphicx}
\usepackage{float}
\usepackage{lipsum}
\usepackage{multicol}
\usepackage{array}
\usepackage{booktabs}
\usepackage{xcolor}
\usepackage{tabularx}
\definecolor{azul}{rgb}{0.0, 0.53, 0.74}
\usepackage[left=2.00cm, right=2.00cm, top=2.00cm, bottom=2.00cm]{geometry}
		    
%%%%%%%%%%%%%%%%%%%%%%%%%%%%%%%%%%%%%%%%%%%%%%%%%%%%%%%%%%%%%%%%%%%%%%%%%%%
\begin{document}	
%%%%%%%%%%%%%%%%%%%%%%%%%%%%%%%%%%%%%%%%%%%%%%%%%%%%%%%%%%%%%%%%%%%%%%%%%%%
	\section{Antecedentes}
		    
%%%%%%%%%%%%%%%%%%%%%%%%%%%%%%%%%%%%%%%%%%%%%%%%%%%%%%%%%%%%%%%%%%%%%%%%%%%
	\subsection{Chatbot para mejorar el proceso de ventas en una empresa de servicios, Lima 2024}

\textbf{Base de Datos (BD):} Los datos para este análisis se recopilaron mediante una ficha de observación que documentó 73 registros de ventas. Estos datos fueron procesados en Google Sheets y posteriormente analizados usando SPSS v.27 para obtener estadísticas descriptivas e inferenciales, asegurando la consistencia y confiabilidad de los resultados. \\

\textbf{Metodología:} Se empleó un enfoque cuantitativo con un diseño de investigación experimental, específicamente pre-experimental. Se aplicaron fichas de observación en pruebas pre-test y post-test a 73 registros del proceso de ventas en una empresa de servicios en Lima, evaluando el impacto de un chatbot en tres dimensiones clave: tiempo de registro de atención, tiempo de elaboración de cotizaciones y confiabilidad del proceso de ventas. La metodología incluye el uso de la prueba de Wilcoxon para validar los resultados y técnicas de observación para garantizar la objetividad de los datos obtenidos. \\

\textbf{Resultados:} La implementación del chatbot en la empresa de servicios en Lima resultó en mejoras significativas en el proceso de ventas: el tiempo de registro de atención se redujo de 71 a 7 segundos, y el tiempo de elaboración de cotizaciones pasó de 130 a 13 segundos, agilizando considerablemente la respuesta al cliente. Además, la confiabilidad del registro de ventas aumentó del 67\% al 99\%, minimizando errores en el proceso. Estos resultados reflejan una optimización en la eficiencia y precisión de la atención al cliente gracias al uso del chatbot. \\

%%%%%%%%%%%%%%%%%%%%%%%%%%%%%%%%%%%%%%%%%%%%%%%%%%%%%%%%%%%%%%%%%%%%%%%%%%%

\subsection{Implementación de un Chatbot Basado en Analítica de Datos para la Disminución de Tiempos de Respuesta y Mejora de la Calidad del Servicio en la Corporación Unificada Nacional de Educación Superior}

\textbf{Base de Datos (BD):} Los datos se recopilan de tres fuentes principales: (1) interacciones de usuarios, donde se analizan las consultas de estudiantes para optimizar las respuestas del chatbot; (2) informes internos de la institución, que brindan información previa sobre tiempos de respuesta y calidad en el servicio de atención; y (3) análisis de métricas mediante Power BI, para visualizar indicadores clave de rendimiento como la satisfacción del usuario y la eficiencia de respuesta. \\

\textbf{Metodología:} El proceso para implementar un dashboard en Power BI utilizando datos de un chatbot en la Universidad CUN incluyó varias etapas clave. Primero, se recolectaron requisitos a través de reuniones con representantes de la universidad para identificar métricas críticas. Luego, se analizaron los datos del chatbot para evaluar su calidad y adecuación. Posteriormente, se exploraron las capacidades de Power BI y se diseñaron prototipos iniciales del dashboard para obtener retroalimentación. A continuación, se desarrolló el dashboard completo, integrando visualizaciones interactivas y configuraciones específicas. Se realizaron pruebas exhaustivas, seguido de entrenamiento al personal en su uso y un despliegue controlado en la universidad. Finalmente, se estableció un plan de mejora continua para optimizar el desempeño del dashboard según el feedback recibido, asegurando que cumpliera los objetivos de visualización y análisis de datos de la universidad. \\

\textbf{Resultados:} Los resultados muestran que la implementación del chatbot CUNDIGITAL mejoró significativamente la calidad de la atención a los estudiantes al reducir los tiempos de respuesta y optimizar la gestión de consultas frecuentes. Con el uso del chatbot y el apoyo de un dashboard en Power BI, la universidad logró visualizar y analizar datos en tiempo real, lo que facilitó la toma de decisiones para ajustes y mejoras continuas en el servicio. Esto resultó en una experiencia más ágil y eficiente para los estudiantes, fortaleciendo la interacción con la institución y mejorando la satisfacción general de los usuarios. \\

%%%%%%%%%%%%%%%%%%%%%%%%%%%%%%%%%%%%%%%%%%%%%%%%%%%%%%%%%%%%%%%%%%%%%%%%%%%

\subsection{La Usabilidad Percibida de los Chatbots sobre la Atención al Cliente en las Organizaciones: Una Revisión de la Literatura}

\textbf{Base de Datos (BD):} Para la recopilación de información, se emplearon términos clave como “Chatbot”, “Agente conversacional” y “Atención al cliente”, realizándose la búsqueda en bases de datos especializadas como EBSCO y Google Académico, utilizando operadores booleanos para optimizar la selección de estudios relevantes. La revisión sistemática abarcó 38 estudios publicados entre 2015 y 2020 que exploran la usabilidad de los chatbots en distintos sectores organizacionales, priorizando investigaciones centradas en la implementación, satisfacción del usuario y efectividad en atención al cliente. \\

\textbf{Metodología:} Este estudio es una revisión sistemática de la literatura científica realizada siguiendo la metodología PRISMA para examinar la usabilidad percibida de los chatbots en la atención al cliente en organizaciones. La revisión sistemática busca sintetizar los resultados de estudios previos, proporcionando un panorama completo sobre el tema y recomendaciones para futuras investigaciones. \\

\textbf{Resultados:} Esta revisión sintetiza los resultados de 38 estudios sobre la usabilidad percibida de los chatbots en la atención al cliente, publicados en Google Académico. Se concluyó que la implementación de chatbots en las organizaciones mejora significativamente la atención al cliente, aunque uno de los principales retos es el análisis de herramientas para su desarrollo y adaptación a los objetivos de cada organización. A medida que se avanza en la comprensión de los chatbots, también surge la necesidad de realizar estudios con diseños más complejos para evaluar el impacto de diferentes tipos de chatbots. Se espera que esta sistematización fomente un mayor número de investigaciones sobre el uso de chatbots y su usabilidad en el servicio de atención al cliente. \\

%%%%%%%%%%%%%%%%%%%%%%%%%%%%%%%%%%%%%%%%%%%%%%%%%%%%%%%%%%%%%%%%%%%%%%%%%%%

\subsection{Diseño de un Chatbot Educativo para la Gestión del Aprendizaje en la Educación Superior}

\textbf{Base de Datos (BD):} Los datos de la tesis provienen de dos fuentes principales: encuestas y entrevistas aplicadas a docentes y estudiantes de la Universidad Mayor de San Andrés (UMSA) en las carreras de Ciencias de la Educación y Ciencias Sociales, que recopilaron información sobre su conocimiento, percepción y disposición para utilizar chatbots en el ámbito educativo; y rúbricas de funcionalidad del chatbot, que se emplearon durante su validación para evaluar aspectos como pertinencia, facilidad de uso, personalización, autenticidad y retroalimentación, aplicándose tras la implementación piloto del chatbot para medir su efectividad en el contexto educativo. \\

\textbf{Metodología:} El estudio emplea un enfoque cuantitativo y un diseño cuasi-experimental, evaluando el chatbot en su funcionalidad y aceptación entre los estudiantes y docentes de la Universidad Mayor de San Andrés, en Bolivia. La investigación se estructura en tres fases: diagnóstico de percepción sobre chatbots, diseño del chatbot basado en las necesidades educativas, y validación de su efectividad mediante encuestas y entrevistas a docentes y estudiantes. \\

\textbf{Resultados:} Los resultados indican una aceptación positiva del chatbot como herramienta de apoyo para los docentes y un recurso útil para los estudiantes en la gestión autónoma del aprendizaje. Los estudiantes valoraron su capacidad para proporcionar respuestas rápidas y relevantes en tiempo real, destacando su potencial para complementar la enseñanza y resolver dudas comunes, aunque se sugirieron mejoras en la personalización y en la calidad de la retroalimentación. \\

%%%%%%%%%%%%%%%%%%%%%%%%%%%%%%%%%%%%%%%%%%%%%%%%%%%%%%%%%%%%%%%%%%%%%%%%%%%

\subsection{Impacto de los Modelos Generativos de Lenguaje de Inteligencia Artificial en la Educación Superior}

\textbf{Base de Datos (BD):} Los datos se obtienen de dos fuentes principales: una revisión bibliográfica de estudios previos en Google Académico, utilizando términos como ChatGPT, inteligencia artificial, modelos generativos de lenguaje y su aplicación en educación superior; y un estudio de caso que incluye una encuesta aplicada a 21 estudiantes de Ingeniería Industrial en el Tecnológico Nacional de México, quienes interactuaron con un chatbot desarrollado en ChatGPT. Esta encuesta, realizada a través de Google Forms, recoge las percepciones de los estudiantes sobre el uso del chatbot en el contexto educativo. \\

\textbf{Metodología:} La metodología incluyó una revisión bibliográfica y un estudio de caso. La búsqueda de literatura se realizó en Google Académico utilizando términos clave como ChatGPT, inteligencia artificial, modelos generativos de lenguaje y ChatGPT en educación superior, complementada con una búsqueda manual en las referencias de los estudios seleccionados. Para el estudio de caso, se aplicó una encuesta a 21 estudiantes de Ingeniería Industrial en el Tecnológico Nacional de México, Campus Pabellón de Arteaga, usando un chatbot diseñado específicamente con ChatGPT a través de la aplicación Poe. La encuesta, distribuida por Google Forms, combinó preguntas en escala de Likert y preguntas dicotómicas, permitiendo un análisis estadístico descriptivo que exploró patrones en las respuestas y garantiza la confidencialidad de los participantes. \\

\textbf{Resultados:} El estudio de caso sobre el uso de ChatGPT en educación superior muestra una aceptación positiva por parte de los estudiantes, quienes destacaron mejoras en su comprensión y en la resolución de dudas. También se observó un aumento en la participación y comunicación en clase, lo cual favorece la interactividad en el aula. La mayoría de los estudiantes percibieron el chatbot como una herramienta útil y eficaz, recomendando su uso en otros cursos. Sin embargo, identificaron posibles mejoras, como la personalización y funciones adicionales, para satisfacer mejor sus necesidades académicas. \\

%%%%%%%%%%%%%%%%%%%%%%%%%%%%%%%%%%%%%%%%%%%%%%%%%%%%%%%%%%%%%%%%%%%%%%%%%%%

\subsection{Chatbot en la Mejora del Proceso de Ventas en la Empresa Newocean Technology S.A.C., Lima 2021}

\textbf{Base de Datos (BD):} Los datos en este caso se obtienen de registros del proceso de ventas, incluyendo tiempos de espera y respuesta en canales como llamadas telefónicas y correos electrónicos. Además, se utilizan métricas antes y después de la implementación para evaluar la reducción en tiempos de cotización y respuesta. La recopilación de datos se realiza a través de observaciones sistemáticas y mediciones de indicadores clave, como tiempos de espera promedio y satisfacción del cliente. \\

\textbf{Metodología:} El estudio es de tipo aplicado, descriptivo y predictivo, enfocado en medir el impacto de un chatbot en el proceso de ventas de una empresa. El diseño es experimental puro, manipulando la variable independiente (presencia del chatbot) y observando su efecto en variables dependientes, como el tiempo promedio de espera, el tiempo para dar respuestas y el tiempo para generar cotizaciones. Se trabajó con una muestra de 50 procesos de venta para cada indicador, utilizando un muestreo aleatorio simple. Los datos fueron recolectados mediante observación y analizados con herramientas estadísticas como Excel y Minitab 18. \\

\textbf{Resultados:} La implementación del chatbot resultó en una reducción drástica en los tiempos de espera y respuesta para la atención al cliente. El tiempo promedio de espera disminuyó de 688,84 segundos a 11,46 segundos, gracias a la capacidad del chatbot para atender múltiples clientes simultáneamente. Asimismo, el tiempo promedio de respuesta pasó de 994,4 segundos a 10,91 segundos. Además, la automatización del proceso de generación de cotizaciones logró reducir el tiempo promedio de 1538,0 segundos a solo 14,25 segundos, mostrando una mejora significativa en la eficiencia del proceso de atención. \\

%%%%%%%%%%%%%%%%%%%%%%%%%%%%%%%%%%%%%%%%%%%%%%%%%%%%%%%%%%%%%%%%%%%%%%%%%%%

\subsection{Implementación de un Chatbot Utilizando Scrum y XP para el Proceso de Atención al Cliente en una Empresa Financiera}

\textbf{Base de Datos (BD):} Los datos se recopilan a partir de observaciones y registros de consultas de usuarios dentro del área de atención al cliente. Esto incluye mediciones de tiempos de respuesta y eficiencia operativa antes y después de implementar el chatbot. Además, se aplican pruebas de aceptación y encuestas de satisfacción para evaluar el impacto del chatbot en la experiencia del cliente. \\

\textbf{Metodología:} El diseño experimental puro, basado en un grupo de pruebas y en datos recolectados estadísticamente, busca identificar conclusiones estableciendo relaciones de causa y efecto a través de la evaluación de resultados. Para evaluar los frameworks de desarrollo de chatbots, se utiliza la técnica AHP, que facilita la toma de decisiones complejas al estructurar y considerar múltiples factores relevantes en las organizaciones. \\

\textbf{Resultados:} Los resultados muestran una mejora significativa en el proceso de atención al cliente. El porcentaje de clientes atendidos dentro de 10 minutos aumentó en un 75\%, mientras que el porcentaje total de atenciones atendidas se incrementó en un 76\%, debido a la mayor capacidad de respuesta. Además, el tiempo promedio de atención se redujo a 28 minutos, lo que contribuyó a una mayor eficiencia. Finalmente, el nivel de satisfacción de los clientes mejoró, alcanzando un 94\% de respuestas satisfactorias por parte del chatbot, lo que elevó la valoración del servicio de "malo" a "muy bueno". \\

%%%%%%%%%%%%%%%%%%%%%%%%%%%%%%%%%%%%%%%%%%%%%%%%%%%%%%%%%%%%%%%%%%%%%%%%%%%

\subsection{Prototipo de Chatbot para la Resolución y Atención de Inquietudes Académicas de la Secretaría de Ingeniería en Sistemas Computacionales e Informáticos}

\textbf{Base de Datos (BD):} En este proyecto, los datos provienen de encuestas realizadas a los estudiantes de la carrera de Ingeniería en Sistemas para identificar las consultas frecuentes. Además, se llevan a cabo entrevistas con el personal de secretaría para definir las respuestas automatizadas más comunes. Los datos también incluyen pruebas de funcionalidad del chatbot para asegurar su rendimiento y precisión. \\

\textbf{Metodología:} La metodología de la investigación se basa en una combinación de enfoques bibliográficos, de campo y experimentales. Se revisaron libros, artículos y tesis para establecer el contexto y la fundamentación teórica. Para recolectar datos, se realizaron encuestas a 50 estudiantes y una entrevista a la secretaria de la carrera para identificar las consultas más frecuentes y los tiempos de respuesta actuales. A partir de estos datos, se diseñó y desarrolló un prototipo de chatbot que fue probado para evaluar su efectividad en la mejora del proceso de atención. La investigación también incluyó un análisis descriptivo e inferencial de los datos recolectados para validar el impacto del chatbot en la eficiencia del servicio. \\

\textbf{Resultados:} Se identificó que la principal razón por la cual los estudiantes solicitan información a la secretaría de la carrera está relacionada con los trámites administrativos. Tras analizar diferentes tecnologías, se optó por IBM Watson Assistant para el desarrollo de un chatbot, debido a su experiencia en inteligencia artificial y flexibilidad para integrarse en múltiples plataformas como web, WhatsApp y Facebook Messenger, sin necesidad de programación avanzada. Las pruebas realizadas confirmaron que el chatbot funciona de acuerdo con los requerimientos establecidos, proporcionando una solución eficiente y accesible para quienes no tienen conocimientos técnicos. \\

%%%%%%%%%%%%%%%%%%%%%%%%%%%%%%%%%%%%%%%%%%%%%%%%%%%%%%%%%%%%%%%%%%%%%%%%%%%

\section{Marco Teórico}

\begin{itemize}
    \item \textbf{Chatbot}
    \item \textbf{Proceso de Ventas}
    \item \textbf{Experiencia del Cliente}
    \item \textbf{Teoría de Colas}
    \item \textbf{Business Intelligence (BI)}
    \item \textbf{Sistemas de Soporte a Decisiones (DSS)}
    \item \textbf{Inteligencia Artificial (IA)}
    \item \textbf{Procesamiento de Lenguaje Natural (PLN)}
\end{itemize}
	
\end{document}

