\documentclass[10pt,a4paper]{article}
\usepackage[utf8]{inputenc}
\usepackage[spanish,es-tabla]{babel}
\usepackage{url}
\usepackage{natbib}
\usepackage{hyperref}
\usepackage{amsmath}
\usepackage{amsfonts}
\usepackage{amssymb}
\usepackage{graphicx}
\usepackage{float}
\usepackage{lipsum}
\usepackage{multicol}
\usepackage{array}
\usepackage{booktabs}
\usepackage{xcolor}
\usepackage{tabularx}
\definecolor{azul}{rgb}{0.0, 0.53, 0.74}
\usepackage[left=2.00cm, right=2.00cm, top=2.00cm, bottom=2.00cm]{geometry}
		    
%%%%%%%%%%%%%%%%%%%%%%%%%%%%%%%%%%%%%%%%%%%%%%%%%%%%%%%%%%%%%%%%%%%%%%%%%%%
\begin{document}	
%%%%%%%%%%%%%%%%%%%%%%%%%%%%%%%%%%%%%%%%%%%%%%%%%%%%%%%%%%%%%%%%%%%%%%%%%%%
	\section{Justificacion de la Investigacion}
		    
%%%%%%%%%%%%%%%%%%%%%%%%%%%%%%%%%%%%%%%%%%%%%%%%%%%%%%%%%%%%%%%%%%%%%%%%%%%
	\subsection{Teorica}

	La investigación tiene como objetivo transformar el proceso de atención al cliente en la venta de cursos de postgrado mediante la implementación de un chatbot basado en inteligencia artificial. El uso de este chatbot permitirá ofrecer respuestas rápidas y personalizadas, optimizando la experiencia del cliente desde el primer contacto hasta la cotización final. Al aplicar metodologías ágiles, el estudio aporta un enfoque novedoso en el ámbito educativo, específicamente en la atención automatizada para la promoción de programas de postgrado. Los resultados serán aplicables a otras instituciones educativas con necesidades similares, proporcionando un marco de referencia para mejorar la competitividad en el mercado de la educación superior.

 
%%%%%%%%%%%%%%%%%%%%%%%%%%%%%%%%%%%%%%%%%%%%%%%%%%%%%%%%%%%%%%%%%%%%%%%%%%

    \subsection{Practica}
	La implementación del chatbot mejorará la eficiencia operativa del proceso de ventas de programas de postgrado, reduciendo tiempos de espera, agilizando respuestas y acelerando la generación de cotizaciones. Esto beneficiará directamente tanto a la institución educativa, que optimizará sus procesos y aumentará su competitividad, como a los potenciales estudiantes, quienes recibirán un servicio más rápido y eficiente. Además, este estudio proporcionará un modelo replicable para otras instituciones educativas que buscan automatizar sus procesos de ventas, reduciendo costos y mejorando la experiencia del cliente.

  
%%%%%%%%%%%%%%%%%%%%%%%%%%%%%%%%%%%%%%%%%%%%%%%%%%%%%%%%%%%%%%%%%%%%%%%%%%

    \subsection{Metodologia}
	Esta investigación aborda problemas clave en el proceso de ventas de cursos de postgrado, como los largos tiempos de espera, la demora en la respuesta al cliente y el tiempo necesario para generar cotizaciones. La automatización de estos procesos mediante el chatbot reducirá significativamente estos tiempos, mejorando la experiencia del cliente. Se utilizarán métodos cualitativos y cuantitativos, incluyendo encuestas a los potenciales clientes y análisis de los tiempos de respuesta actuales, para medir la eficacia del chatbot. Además, permitirá segmentar mejor a los prospectos, ofreciendo soluciones más personalizadas que optimizan los tiempos de atención y mejoren la calidad de la interacción en el proceso de ventas.

 %%%%%%%%%%%%%%%%%%%%%%%%%%%%%%%%%%%%%%%%%%%%%%%%%%%%%%%%%%%%%%%%%%%%%%%%%%%
	
\end{document}

